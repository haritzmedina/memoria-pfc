\chapter{Conclusiones}
\label{cha:conclusiones}

Para concluir se presentan ciertas conclusiones generales relacionadas con el PFG y la herramienta desarrollada (Apartado \ref{sec:ConclusionesGenerales}). A su vez, posibles futuras líneas de trabajo derivadas de las limitaciones actuales del propio proyecto (Apartado \ref{sec:TrabajoFuturo}). Por último, unas lecciones aprendidas a lo largo del PFG (Apartado \ref{sec:Lecciones}).

\section{Conclusiones generales}
\label{sec:ConclusionesGenerales}

El uso de las tecnologías web para desarrollar aplicaciones está sustituyendo en gran medida a las aplicaciones tradicionales de escritorio. La razón principal es que no requiere de ninguna instalación (aparte del navegador) y que es accesible de manera global, permitiendo también un trabajo colaborativo.

Dado que desarrollar software requiere de gran conocimiento y tiempo y no todos disponen de ello, desarrollar herramientas DIY permite:
\begin{itemize}
\item{El usuario sea el responsable del desarrollo que va a consumir posteriormente.}
\item{Gracias al DIY el usuario puede cumplir con sus propias especificaciones y valorar de manera más precisa si logra satisfacer sus necesidades.}
\item{Le permite modificar su desarrollo de manera sencilla, ya que conoce todos los aspectos del mismo.}
\end{itemize}

Por tanto, la utilidad de herramientas de este estilo, puede hacer que se generen aumentaciones útiles. Esto hace que la gente pueda ahorrar mucho tiempo en algunas tareas y ser, por un lado, más productivos, y por otro lado, desempeñar mejor la labor.

Cabe destacar, que el proyecto tiene ciertos valores positivos a destacar a nivel global.

El realizar este PFG ha permitido poder completar algunas funcionalidades de la herramienta WebMakeUp. Con ello se permite investigar sobre las aumentaciones web del lado cliente. Esto es muy provechoso para el grupo Onekin, que ha decidido utilizar algunas de las ideas aquí desarrolladas para la creación de 2 artículos de investigación que se han enviado a dos congresos. 

El primer artículo \cite{ICWEWebMakeUp} se envía al congreso ICWE 2014\footnote{Sitio web oficial del congreso ICWE 2014: \url{http://icwe2014.webengineering.org/}}, donde finalmente no se acepta. En él se muestra la primera versión de WebMakeUp, donde las animaciones se hacen en el modelo basado en STDs.

El segundo artículo \cite{WISEWebMakeUp} se envía al congreso WISE 2014 de Doha\footnote{Sitio web oficial de la conferencia WISE: \url{http://www.wise-qatar.org/}}, que aún está pendiente de ser aceptado. En él se muestra cómo es la segunda versión de WebMakeUp, donde se mejoraron muchos aspectos. Entre ellos, el cambio al modelo basado en blinks.

\section{Líneas de trabajo futuras}
\label{sec:TrabajoFuturo}

El desarrollo realizado en este PFG tiene una limitación muy clara. La limitación es que sólo te permite atender la demanda de los usuarios de Google Chrome.

Por lo tanto, una de las líneas de trabajo que se podría trabajar en el futuro son:
\begin{itemize}
\item{\textbf{Desarrollar diferentes generadores} (uno por cada navegador o al menos abarcar la de la mayoría). Esto implicaría una labor muy tediosa por dos aspectos principales. Uno, es necesario conocer cómo es la plataforma específica de desarrollo para cada uno de los navegadores, dado que no existe una plataforma común. Dos, requiere de un desarrollo para cada uno de los navegadores. Esto implica una cantidad ingente de horas con tal de cubrir las necesidades de al menos los principales navegadores (Internet Explorer, Opera, Safari, Firefox y algún navegador móvil como Dolphin\footnote{Sitio web de Dolphin Browser: \url{http://dolphin.com/}.}.}
\item{\textbf{Desarrollar un generador} común a todas las plataformas. Para ello existen maneras de aumentar la web mediante el uso de \emph{userscripts}. Los userscripts son compatibles con diferentes navegadores (al menos Chrome y Firefox ya los soportan de manera nativa, y otros navegadores mediante extensiones interprete). Esto implica que una misma extensión/script generada sería compatible con diferentes navegadores.}
\end{itemize}

\section{Lecciones aprendidas}
\label{sec:Lecciones}

A lo largo del PFG hay muchas lecciones aprendidas, relacionadas con muchos aspectos, no sólo del desarrollo, si no también con la investigación, la gestión, el trabajo en equipo, y un largo etcétera. Aquí se describen las más relevantes:

\begin{itemize}
\item{El desarrollo para usuarios finales requiere de pruebas con ellos. A pesar de resultar trivial, en este proyecto se refleja que este es uno de los conceptos más importantes. Un usuario final muestra sus necesidades, ideas, metodologías de trabajo. Muchas veces se puede desarrollar una herramienta muy potente pero que nadie es capaz de usarla. Detectar esto es complejo si no se realizan \emph{tests} con usuarios finales.}
\item{El desarrollo basado en metodologías ágiles es muy adecuado para construir herramientas innovadoras. El ir implementando a medida que se van obteniendo nuevas ideas, ayuda a ir descartándolas o aceptándolas sin tener que desarrollarlas completamente.}
\item{El análisis es muy importante en líneas de investigación. Cuando se dispone de una idea innovadora, hay que buscar información para garantizar que se puede llevar a cabo y para cerciorarse de que no se está reinventando la rueda. Ser hábil buscando información relacionada es importante, dado que permite no dedicar demasiado tiempo a tareas que ya han sido desarrolladas previamente, pudiendo dedicar más tiempo a la parte no trillada.}
\item{El seguimiento y control continuo es necesario en los proyectos donde no se tiene definido un objetivo claro. Permite ir viendo si se alcanzan o no los objetivos, y evita generar sobre-costes en partes no fundamentales de la investigación.}
\end{itemize}