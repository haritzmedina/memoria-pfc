\chapter{Documento de Objetivos del Proyecto}
\label{cha:DOP}

% Cabecera de pagina con DOP
\chaptermark{DOP}

Hay diversos objetivos en este proyecto. Por un lado, la participación en el proyecto WebMakeUp con la realización de tres tareas. Por otro lado, a nivel personal la realización de un proyecto con el que aprender a utilizar diferentes tecnologías. Estas tecnologías servirán tanto para la realización del producto como del proyecto.

Estas tres tareas se han ido definiendo a medida que el proyecto iba progresando, dado que al ser un proyecto de investigación, el objetivo va cambiando a medida que se va determinando por dónde avanzar. Por tanto se podría decir que en un inicio los objetivos no estaban claramente definidos y se han ido encaminando durante el transcurso del mismo. De igual manera, al tratarse de un desarrollo ágil no toda la funcionalidad desarrollada se ha puesto en marcha, si no que se han podido descartar partes a lo largo del propio desarrollo.

La primera de las tres tareas de desarrollo viene dada en los comienzos de WebMakeUp. En la primera fase del desarrollo, se requiere una labor de análisis del contexto. Hay que buscar una manera sencilla de cara al usuario de representar la zona de trabajo y el cómo se van a disponer de las herramientas para que fuese \emph{user-friendly} (sencillo para el usuario final). Para ello se tuvo que hacer un estudio de editores de diferente tipo con tal de encontrar alguno que pudiera coincidir con lo que se quiere describir en WebMakeUp, una aumentación web.

La segunda de las tareas del proyecto viene a partir de la necesidad de trabajar con las interacciones entre widgets. Un usuario, al interaccionar con el sitio web, genera eventos. Estos producen cambios de contenido. En primer lugar, se ha trabajado con diagramas de transición de estados (STD). Cada uno de estos estados representa una disposición de los widgets y las transiciones entre estados se hacen en base a eventos (Apartado \ref{sec:modeloSTD}). Posteriormente tras pruebas con usuarios finales se decide cambiar de representación y utilizar un sistema basado en blinks (Apartado \ref{sec:modeloBlinks}).

La tercera de las tareas y la más importante una vez teniendo un primer prototipo funcional de la herramienta es reflejar esa aumentación. A partir de un modelo de datos que proporciona el editor generar una extensión de Google Chrome con la aumentación descrita. Por tanto se trata de un desarrollo en base a modelos descritos en un lenguaje específico de dominio (DSL). Para ello se definen algoritmos y técnicas que hay que verificar meticulosamente.

Al igual que como objetivos se plantean estas tareas, también hay otros objetivos transversales en la realización de este proyecto. Entre ellos destacan: el aprendizaje de cómo funciona un proyecto de investigación, el trabajar con documentación colaborativa, desarrollo con control de versiones (en este caso GIT) y metodologías ágiles basada en prototipos, gestión de tareas mediante una herramienta colaborativa, gestión de la calidad en el desarrollo, aprendizaje de patrones de diseño, etc. En general, diferentes técnicas que complementan un proyecto innovador.

Otro de los posibles objetivos que se podía haber desarrollado en este proyecto es el desarrollo multiplataforma de las aumentaciones realizadas. Esto implica disponer de múltiples generadores, uno por cada navegador al que se quiera extender estas aumentaciones; o disponer de un tipo de extensión compatible con todas las plataformas, como por ejemplo los \emph{userscripts}\footnote{Descripción de userscripts: http://en.wikipedia.org/wiki/Userscript}.