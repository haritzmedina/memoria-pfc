\chapter*{\abstract}
\addcontentsline{toc}{chapter}{\abstract}
\setcounter{page}{1}
% \thispagestyle{empty}

No todos los usuarios interaccionan de la misma manera con el mismo sitio web. Tienen diferentes necesidades, objetivos, costumbres y conocimientos. Esto suele derivar en una navegación en la web bastante pobre. 

Una navegación pobre hace que el usuario deje de ser productivo. No consigue focalizarse en la tarea que quiere realizar, dedica excesivo tiempo, o incluso, es incapaz de realizarla.

De ahí la necesidad de hacer adaptaciones de los sitios web a diferentes usuarios. Por desgracia, el desarrollador del sitio puede carecer del tiempo, los recursos, o la visión para soportar esta personalización. Esto avala el interés de que sean los propios usuarios quienes adapten el sitio web. El principal inconveniente de esto, es los conocimientos que se le exigen al usuario para realizar esta adaptación.

Este proyecto se enmarca en el desarrollo de un editor gráfico para soportar la adaptación por parte de usuarios finales. A este tipo de técnica se le conoce como \emph{Web Augmentation} (a partir de ahora WA).

El termino WA hace referencia a cambios en un sitio web ofreciendo una realidad (el sitio web original) pero aumentada (con modificaciones hechas por el usuario sobre esa realidad), con las que este puede ser más productivo o sentirse más cómodo.

En este proyecto se han trabajado en tres áreas. El primer área, relacionado con la investigación sobre editores utilizados en el mercado. El segundo, relacionado con el desarrollo de animaciones de widgets de un sitio web concreto. Finalmente, el tercero, orientado a la generación automática de una WA a partir de un modelo generado por WebMakeUp, un editor de WAs para usuarios finales.

\textbf{Palabras clave}: Aumentación web, Programación de usuario final, Javascript